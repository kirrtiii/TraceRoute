%% Generated by Sphinx.
\def\sphinxdocclass{report}
\documentclass[letterpaper,10pt,english]{sphinxmanual}
\ifdefined\pdfpxdimen
   \let\sphinxpxdimen\pdfpxdimen\else\newdimen\sphinxpxdimen
\fi \sphinxpxdimen=.75bp\relax
\ifdefined\pdfimageresolution
    \pdfimageresolution= \numexpr \dimexpr1in\relax/\sphinxpxdimen\relax
\fi
%% let collapsible pdf bookmarks panel have high depth per default
\PassOptionsToPackage{bookmarksdepth=5}{hyperref}

\PassOptionsToPackage{booktabs}{sphinx}
\PassOptionsToPackage{colorrows}{sphinx}

\PassOptionsToPackage{warn}{textcomp}
\usepackage[utf8]{inputenc}
\ifdefined\DeclareUnicodeCharacter
% support both utf8 and utf8x syntaxes
  \ifdefined\DeclareUnicodeCharacterAsOptional
    \def\sphinxDUC#1{\DeclareUnicodeCharacter{"#1}}
  \else
    \let\sphinxDUC\DeclareUnicodeCharacter
  \fi
  \sphinxDUC{00A0}{\nobreakspace}
  \sphinxDUC{2500}{\sphinxunichar{2500}}
  \sphinxDUC{2502}{\sphinxunichar{2502}}
  \sphinxDUC{2514}{\sphinxunichar{2514}}
  \sphinxDUC{251C}{\sphinxunichar{251C}}
  \sphinxDUC{2572}{\textbackslash}
\fi
\usepackage{cmap}
\usepackage[T1]{fontenc}
\usepackage{amsmath,amssymb,amstext}
\usepackage{babel}



\usepackage{tgtermes}
\usepackage{tgheros}
\renewcommand{\ttdefault}{txtt}



\usepackage[Bjarne]{fncychap}
\usepackage{sphinx}

\fvset{fontsize=auto}
\usepackage{geometry}


% Include hyperref last.
\usepackage{hyperref}
% Fix anchor placement for figures with captions.
\usepackage{hypcap}% it must be loaded after hyperref.
% Set up styles of URL: it should be placed after hyperref.
\urlstyle{same}

\addto\captionsenglish{\renewcommand{\contentsname}{Contents:}}

\usepackage{sphinxmessages}
\setcounter{tocdepth}{1}



\title{Ping and traceroute}
\date{Feb 17, 2025}
\release{}
\author{Kirti Sharma}
\newcommand{\sphinxlogo}{\vbox{}}
\renewcommand{\releasename}{}
\makeindex
\begin{document}

\ifdefined\shorthandoff
  \ifnum\catcode`\=\string=\active\shorthandoff{=}\fi
  \ifnum\catcode`\"=\active\shorthandoff{"}\fi
\fi

\pagestyle{empty}
\sphinxmaketitle
\pagestyle{plain}
\sphinxtableofcontents
\pagestyle{normal}
\phantomsection\label{\detokenize{index::doc}}


\sphinxstepscope


\chapter{ping module}
\label{\detokenize{ping:module-ping}}\label{\detokenize{ping:ping-module}}\label{\detokenize{ping::doc}}\index{module@\spxentry{module}!ping@\spxentry{ping}}\index{ping@\spxentry{ping}!module@\spxentry{module}}
\sphinxAtStartPar
Python implementation of the ping utility.

\sphinxAtStartPar
This module provides functionality to ping a specified host.
\index{ping() (in module ping)@\spxentry{ping()}\spxextra{in module ping}}

\begin{fulllineitems}
\phantomsection\label{\detokenize{ping:ping.ping}}
\pysigstartsignatures
\pysiglinewithargsret
{\sphinxcode{\sphinxupquote{ping.}}\sphinxbfcode{\sphinxupquote{ping}}}
{\sphinxparam{\DUrole{n}{host}}\sphinxparamcomma \sphinxparam{\DUrole{n}{count}}\sphinxparamcomma \sphinxparam{\DUrole{n}{interval}}\sphinxparamcomma \sphinxparam{\DUrole{n}{packet\_size}}\sphinxparamcomma \sphinxparam{\DUrole{n}{timeout}}}
{}
\pysigstopsignatures
\sphinxAtStartPar
Ping a specified host.

\sphinxAtStartPar
This function sends ICMP Echo Request packets to a specified host and
reports on the round\sphinxhyphen{}trip time and packet loss statistics.
\begin{quote}\begin{description}
\sphinxlineitem{Parameters}\begin{itemize}
\item {} 
\sphinxAtStartPar
\sphinxstyleliteralstrong{\sphinxupquote{host}} (\sphinxstyleliteralemphasis{\sphinxupquote{str}}) \textendash{} Hostname or IP address to ping

\item {} 
\sphinxAtStartPar
\sphinxstyleliteralstrong{\sphinxupquote{count}} (\sphinxstyleliteralemphasis{\sphinxupquote{int}}) \textendash{} Number of ping requests to send (0 for infinite)

\item {} 
\sphinxAtStartPar
\sphinxstyleliteralstrong{\sphinxupquote{interval}} (\sphinxstyleliteralemphasis{\sphinxupquote{float}}) \textendash{} Time interval between ping requests in seconds

\item {} 
\sphinxAtStartPar
\sphinxstyleliteralstrong{\sphinxupquote{packet\_size}} (\sphinxstyleliteralemphasis{\sphinxupquote{int}}) \textendash{} Size of the ping packet in bytes

\item {} 
\sphinxAtStartPar
\sphinxstyleliteralstrong{\sphinxupquote{timeout}} (\sphinxstyleliteralemphasis{\sphinxupquote{float}}) \textendash{} Timeout for each ping request in seconds

\end{itemize}

\end{description}\end{quote}

\end{fulllineitems}


\sphinxstepscope


\chapter{traceroute module}
\label{\detokenize{traceroute:module-traceroute}}\label{\detokenize{traceroute:traceroute-module}}\label{\detokenize{traceroute::doc}}\index{module@\spxentry{module}!traceroute@\spxentry{traceroute}}\index{traceroute@\spxentry{traceroute}!module@\spxentry{module}}
\sphinxAtStartPar
Python implementation of the traceroute utility.

\sphinxAtStartPar
This module provides functionality to trace the route to a specified host,
showing the path that packets take to reach the destination.
\index{traceroute() (in module traceroute)@\spxentry{traceroute()}\spxextra{in module traceroute}}

\begin{fulllineitems}
\phantomsection\label{\detokenize{traceroute:traceroute.traceroute}}
\pysigstartsignatures
\pysiglinewithargsret
{\sphinxcode{\sphinxupquote{traceroute.}}\sphinxbfcode{\sphinxupquote{traceroute}}}
{\sphinxparam{\DUrole{n}{dest\_addr}}\sphinxparamcomma \sphinxparam{\DUrole{n}{max\_hops}\DUrole{o}{=}\DUrole{default_value}{30}}\sphinxparamcomma \sphinxparam{\DUrole{n}{timeout}\DUrole{o}{=}\DUrole{default_value}{1}}\sphinxparamcomma \sphinxparam{\DUrole{n}{queries}\DUrole{o}{=}\DUrole{default_value}{3}}\sphinxparamcomma \sphinxparam{\DUrole{n}{numeric}\DUrole{o}{=}\DUrole{default_value}{False}}\sphinxparamcomma \sphinxparam{\DUrole{n}{summary}\DUrole{o}{=}\DUrole{default_value}{False}}}
{}
\pysigstopsignatures
\sphinxAtStartPar
Perform a traceroute to a specified destination.

\sphinxAtStartPar
This function sends packets with increasing TTL values to discover the path
to the destination and measure round\sphinxhyphen{}trip times for each hop.
\begin{quote}\begin{description}
\sphinxlineitem{Parameters}\begin{itemize}
\item {} 
\sphinxAtStartPar
\sphinxstyleliteralstrong{\sphinxupquote{dest\_addr}} (\sphinxstyleliteralemphasis{\sphinxupquote{str}}) \textendash{} Destination hostname or IP address

\item {} 
\sphinxAtStartPar
\sphinxstyleliteralstrong{\sphinxupquote{max\_hops}} (\sphinxstyleliteralemphasis{\sphinxupquote{int}}) \textendash{} Maximum number of hops to probe (default: 30)

\item {} 
\sphinxAtStartPar
\sphinxstyleliteralstrong{\sphinxupquote{timeout}} (\sphinxstyleliteralemphasis{\sphinxupquote{float}}) \textendash{} Timeout for each probe in seconds (default: 1)

\item {} 
\sphinxAtStartPar
\sphinxstyleliteralstrong{\sphinxupquote{queries}} (\sphinxstyleliteralemphasis{\sphinxupquote{int}}) \textendash{} Number of queries per hop (default: 3)

\item {} 
\sphinxAtStartPar
\sphinxstyleliteralstrong{\sphinxupquote{numeric}} (\sphinxstyleliteralemphasis{\sphinxupquote{bool}}) \textendash{} If True, print numeric addresses only (default: False)

\item {} 
\sphinxAtStartPar
\sphinxstyleliteralstrong{\sphinxupquote{summary}} (\sphinxstyleliteralemphasis{\sphinxupquote{bool}}) \textendash{} If True, print summary of unanswered probes (default: False)

\end{itemize}

\end{description}\end{quote}

\end{fulllineitems}


\sphinxstepscope


\chapter{utils module}
\label{\detokenize{utils:module-utils}}\label{\detokenize{utils:utils-module}}\label{\detokenize{utils::doc}}\index{module@\spxentry{module}!utils@\spxentry{utils}}\index{utils@\spxentry{utils}!module@\spxentry{module}}
\sphinxAtStartPar
Utility functions for network operations.

\sphinxAtStartPar
This module provides common functions used in ping and traceroute implementations,
including packet creation, checksum calculation, and DNS resolution.
\index{calculate\_checksum() (in module utils)@\spxentry{calculate\_checksum()}\spxextra{in module utils}}

\begin{fulllineitems}
\phantomsection\label{\detokenize{utils:utils.calculate_checksum}}
\pysigstartsignatures
\pysiglinewithargsret
{\sphinxcode{\sphinxupquote{utils.}}\sphinxbfcode{\sphinxupquote{calculate\_checksum}}}
{\sphinxparam{\DUrole{n}{data}}}
{}
\pysigstopsignatures
\sphinxAtStartPar
Calculate the checksum for an ICMP packet.
\begin{quote}\begin{description}
\sphinxlineitem{Parameters}
\sphinxAtStartPar
\sphinxstyleliteralstrong{\sphinxupquote{data}} (\sphinxstyleliteralemphasis{\sphinxupquote{bytes}}) \textendash{} Data to calculate checksum for

\sphinxlineitem{Returns}
\sphinxAtStartPar
Calculated checksum

\sphinxlineitem{Return type}
\sphinxAtStartPar
int

\end{description}\end{quote}

\end{fulllineitems}

\index{create\_packet() (in module utils)@\spxentry{create\_packet()}\spxextra{in module utils}}

\begin{fulllineitems}
\phantomsection\label{\detokenize{utils:utils.create_packet}}
\pysigstartsignatures
\pysiglinewithargsret
{\sphinxcode{\sphinxupquote{utils.}}\sphinxbfcode{\sphinxupquote{create\_packet}}}
{\sphinxparam{\DUrole{n}{id}}\sphinxparamcomma \sphinxparam{\DUrole{n}{seq}}\sphinxparamcomma \sphinxparam{\DUrole{n}{payload\_size}}}
{}
\pysigstopsignatures
\sphinxAtStartPar
Create an ICMP Echo Request packet.
\begin{quote}\begin{description}
\sphinxlineitem{Parameters}\begin{itemize}
\item {} 
\sphinxAtStartPar
\sphinxstyleliteralstrong{\sphinxupquote{id}} (\sphinxstyleliteralemphasis{\sphinxupquote{int}}) \textendash{} Identifier for the packet

\item {} 
\sphinxAtStartPar
\sphinxstyleliteralstrong{\sphinxupquote{seq}} (\sphinxstyleliteralemphasis{\sphinxupquote{int}}) \textendash{} Sequence number for the packet

\item {} 
\sphinxAtStartPar
\sphinxstyleliteralstrong{\sphinxupquote{payload\_size}} (\sphinxstyleliteralemphasis{\sphinxupquote{int}}) \textendash{} Size of the packet payload

\end{itemize}

\sphinxlineitem{Returns}
\sphinxAtStartPar
Bytes object representing the ICMP packet

\sphinxlineitem{Return type}
\sphinxAtStartPar
bytes

\end{description}\end{quote}

\end{fulllineitems}

\index{get\_hostname() (in module utils)@\spxentry{get\_hostname()}\spxextra{in module utils}}

\begin{fulllineitems}
\phantomsection\label{\detokenize{utils:utils.get_hostname}}
\pysigstartsignatures
\pysiglinewithargsret
{\sphinxcode{\sphinxupquote{utils.}}\sphinxbfcode{\sphinxupquote{get\_hostname}}}
{\sphinxparam{\DUrole{n}{ip\_address}}}
{}
\pysigstopsignatures
\sphinxAtStartPar
Get the hostname for a given IP address.
\begin{quote}\begin{description}
\sphinxlineitem{Parameters}
\sphinxAtStartPar
\sphinxstyleliteralstrong{\sphinxupquote{ip\_address}} (\sphinxstyleliteralemphasis{\sphinxupquote{str}}) \textendash{} IP address to lookup

\sphinxlineitem{Returns}
\sphinxAtStartPar
Hostname if found, otherwise the original IP address

\sphinxlineitem{Return type}
\sphinxAtStartPar
str

\end{description}\end{quote}

\end{fulllineitems}

\index{resolve\_hostname() (in module utils)@\spxentry{resolve\_hostname()}\spxextra{in module utils}}

\begin{fulllineitems}
\phantomsection\label{\detokenize{utils:utils.resolve_hostname}}
\pysigstartsignatures
\pysiglinewithargsret
{\sphinxcode{\sphinxupquote{utils.}}\sphinxbfcode{\sphinxupquote{resolve\_hostname}}}
{\sphinxparam{\DUrole{n}{hostname}}}
{}
\pysigstopsignatures
\sphinxAtStartPar
Resolve a hostname to its IP address.
\begin{quote}\begin{description}
\sphinxlineitem{Parameters}
\sphinxAtStartPar
\sphinxstyleliteralstrong{\sphinxupquote{hostname}} (\sphinxstyleliteralemphasis{\sphinxupquote{str}}) \textendash{} Hostname to resolve

\sphinxlineitem{Returns}
\sphinxAtStartPar
IP address of the hostname, or None if unresolvable

\sphinxlineitem{Return type}
\sphinxAtStartPar
str or None

\end{description}\end{quote}

\end{fulllineitems}

\index{setup\_socket() (in module utils)@\spxentry{setup\_socket()}\spxextra{in module utils}}

\begin{fulllineitems}
\phantomsection\label{\detokenize{utils:utils.setup_socket}}
\pysigstartsignatures
\pysiglinewithargsret
{\sphinxcode{\sphinxupquote{utils.}}\sphinxbfcode{\sphinxupquote{setup\_socket}}}
{}
{}
\pysigstopsignatures
\sphinxAtStartPar
Set up a raw socket for ICMP communication.
\begin{quote}\begin{description}
\sphinxlineitem{Returns}
\sphinxAtStartPar
Configured socket object

\sphinxlineitem{Return type}
\sphinxAtStartPar
socket.socket

\end{description}\end{quote}

\end{fulllineitems}



\chapter{Indices and tables}
\label{\detokenize{index:indices-and-tables}}\begin{itemize}
\item {} 
\sphinxAtStartPar
\DUrole{xref}{\DUrole{std}{\DUrole{std-ref}{genindex}}}

\item {} 
\sphinxAtStartPar
\DUrole{xref}{\DUrole{std}{\DUrole{std-ref}{modindex}}}

\item {} 
\sphinxAtStartPar
\DUrole{xref}{\DUrole{std}{\DUrole{std-ref}{search}}}

\end{itemize}


\renewcommand{\indexname}{Python Module Index}
\begin{sphinxtheindex}
\let\bigletter\sphinxstyleindexlettergroup
\bigletter{p}
\item\relax\sphinxstyleindexentry{ping}\sphinxstyleindexpageref{ping:\detokenize{module-ping}}
\indexspace
\bigletter{t}
\item\relax\sphinxstyleindexentry{traceroute}\sphinxstyleindexpageref{traceroute:\detokenize{module-traceroute}}
\indexspace
\bigletter{u}
\item\relax\sphinxstyleindexentry{utils}\sphinxstyleindexpageref{utils:\detokenize{module-utils}}
\end{sphinxtheindex}

\renewcommand{\indexname}{Index}
\printindex
\end{document}